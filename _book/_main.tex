% Options for packages loaded elsewhere
\PassOptionsToPackage{unicode}{hyperref}
\PassOptionsToPackage{hyphens}{url}
%
\documentclass[
]{book}
\usepackage{lmodern}
\usepackage{amssymb,amsmath}
\usepackage{ifxetex,ifluatex}
\ifnum 0\ifxetex 1\fi\ifluatex 1\fi=0 % if pdftex
  \usepackage[T1]{fontenc}
  \usepackage[utf8]{inputenc}
  \usepackage{textcomp} % provide euro and other symbols
\else % if luatex or xetex
  \usepackage{unicode-math}
  \defaultfontfeatures{Scale=MatchLowercase}
  \defaultfontfeatures[\rmfamily]{Ligatures=TeX,Scale=1}
\fi
% Use upquote if available, for straight quotes in verbatim environments
\IfFileExists{upquote.sty}{\usepackage{upquote}}{}
\IfFileExists{microtype.sty}{% use microtype if available
  \usepackage[]{microtype}
  \UseMicrotypeSet[protrusion]{basicmath} % disable protrusion for tt fonts
}{}
\makeatletter
\@ifundefined{KOMAClassName}{% if non-KOMA class
  \IfFileExists{parskip.sty}{%
    \usepackage{parskip}
  }{% else
    \setlength{\parindent}{0pt}
    \setlength{\parskip}{6pt plus 2pt minus 1pt}}
}{% if KOMA class
  \KOMAoptions{parskip=half}}
\makeatother
\usepackage{xcolor}
\IfFileExists{xurl.sty}{\usepackage{xurl}}{} % add URL line breaks if available
\IfFileExists{bookmark.sty}{\usepackage{bookmark}}{\usepackage{hyperref}}
\hypersetup{
  pdftitle={PHIL 240 CONTEMPORARY MORAL PROBLEMS (ETHICS) (IAI: H4 904)},
  pdfauthor={Montaque Reynolds},
  hidelinks,
  pdfcreator={LaTeX via pandoc}}
\urlstyle{same} % disable monospaced font for URLs
\usepackage{color}
\usepackage{fancyvrb}
\newcommand{\VerbBar}{|}
\newcommand{\VERB}{\Verb[commandchars=\\\{\}]}
\DefineVerbatimEnvironment{Highlighting}{Verbatim}{commandchars=\\\{\}}
% Add ',fontsize=\small' for more characters per line
\usepackage{framed}
\definecolor{shadecolor}{RGB}{248,248,248}
\newenvironment{Shaded}{\begin{snugshade}}{\end{snugshade}}
\newcommand{\AlertTok}[1]{\textcolor[rgb]{0.94,0.16,0.16}{#1}}
\newcommand{\AnnotationTok}[1]{\textcolor[rgb]{0.56,0.35,0.01}{\textbf{\textit{#1}}}}
\newcommand{\AttributeTok}[1]{\textcolor[rgb]{0.77,0.63,0.00}{#1}}
\newcommand{\BaseNTok}[1]{\textcolor[rgb]{0.00,0.00,0.81}{#1}}
\newcommand{\BuiltInTok}[1]{#1}
\newcommand{\CharTok}[1]{\textcolor[rgb]{0.31,0.60,0.02}{#1}}
\newcommand{\CommentTok}[1]{\textcolor[rgb]{0.56,0.35,0.01}{\textit{#1}}}
\newcommand{\CommentVarTok}[1]{\textcolor[rgb]{0.56,0.35,0.01}{\textbf{\textit{#1}}}}
\newcommand{\ConstantTok}[1]{\textcolor[rgb]{0.00,0.00,0.00}{#1}}
\newcommand{\ControlFlowTok}[1]{\textcolor[rgb]{0.13,0.29,0.53}{\textbf{#1}}}
\newcommand{\DataTypeTok}[1]{\textcolor[rgb]{0.13,0.29,0.53}{#1}}
\newcommand{\DecValTok}[1]{\textcolor[rgb]{0.00,0.00,0.81}{#1}}
\newcommand{\DocumentationTok}[1]{\textcolor[rgb]{0.56,0.35,0.01}{\textbf{\textit{#1}}}}
\newcommand{\ErrorTok}[1]{\textcolor[rgb]{0.64,0.00,0.00}{\textbf{#1}}}
\newcommand{\ExtensionTok}[1]{#1}
\newcommand{\FloatTok}[1]{\textcolor[rgb]{0.00,0.00,0.81}{#1}}
\newcommand{\FunctionTok}[1]{\textcolor[rgb]{0.00,0.00,0.00}{#1}}
\newcommand{\ImportTok}[1]{#1}
\newcommand{\InformationTok}[1]{\textcolor[rgb]{0.56,0.35,0.01}{\textbf{\textit{#1}}}}
\newcommand{\KeywordTok}[1]{\textcolor[rgb]{0.13,0.29,0.53}{\textbf{#1}}}
\newcommand{\NormalTok}[1]{#1}
\newcommand{\OperatorTok}[1]{\textcolor[rgb]{0.81,0.36,0.00}{\textbf{#1}}}
\newcommand{\OtherTok}[1]{\textcolor[rgb]{0.56,0.35,0.01}{#1}}
\newcommand{\PreprocessorTok}[1]{\textcolor[rgb]{0.56,0.35,0.01}{\textit{#1}}}
\newcommand{\RegionMarkerTok}[1]{#1}
\newcommand{\SpecialCharTok}[1]{\textcolor[rgb]{0.00,0.00,0.00}{#1}}
\newcommand{\SpecialStringTok}[1]{\textcolor[rgb]{0.31,0.60,0.02}{#1}}
\newcommand{\StringTok}[1]{\textcolor[rgb]{0.31,0.60,0.02}{#1}}
\newcommand{\VariableTok}[1]{\textcolor[rgb]{0.00,0.00,0.00}{#1}}
\newcommand{\VerbatimStringTok}[1]{\textcolor[rgb]{0.31,0.60,0.02}{#1}}
\newcommand{\WarningTok}[1]{\textcolor[rgb]{0.56,0.35,0.01}{\textbf{\textit{#1}}}}
\usepackage{longtable,booktabs}
% Correct order of tables after \paragraph or \subparagraph
\usepackage{etoolbox}
\makeatletter
\patchcmd\longtable{\par}{\if@noskipsec\mbox{}\fi\par}{}{}
\makeatother
% Allow footnotes in longtable head/foot
\IfFileExists{footnotehyper.sty}{\usepackage{footnotehyper}}{\usepackage{footnote}}
\makesavenoteenv{longtable}
\usepackage{graphicx}
\makeatletter
\def\maxwidth{\ifdim\Gin@nat@width>\linewidth\linewidth\else\Gin@nat@width\fi}
\def\maxheight{\ifdim\Gin@nat@height>\textheight\textheight\else\Gin@nat@height\fi}
\makeatother
% Scale images if necessary, so that they will not overflow the page
% margins by default, and it is still possible to overwrite the defaults
% using explicit options in \includegraphics[width, height, ...]{}
\setkeys{Gin}{width=\maxwidth,height=\maxheight,keepaspectratio}
% Set default figure placement to htbp
\makeatletter
\def\fps@figure{htbp}
\makeatother
\setlength{\emergencystretch}{3em} % prevent overfull lines
\providecommand{\tightlist}{%
  \setlength{\itemsep}{0pt}\setlength{\parskip}{0pt}}
\setcounter{secnumdepth}{5}
\usepackage{booktabs}
\usepackage[]{natbib}
\bibliographystyle{plainnat}

\title{PHIL 240 CONTEMPORARY MORAL PROBLEMS (ETHICS) (IAI: H4 904)}
\author{Montaque Reynolds}
\date{2024-08-26}

\begin{document}
\maketitle

{
\setcounter{tocdepth}{1}
\tableofcontents
}
Surveys the major types of ethical theories, such as consequentialist, non-consequentialist, and virtue-based theories, and applies these to a number of contemporary moral controversies. These
controversies include (but are not limited to) abortion, euthanasia, capital punishment, healthcare, sexual morality, professional and business ethics, and the environment. (PCS 1.1, 3 credit hours: 3 hours lecture, 0 hours lab)

\hypertarget{course-information}{%
\chapter{Course Information}\label{course-information}}

Course Number: PHI 240 CONTEMPORARY MORAL PROBLEMS (ETHICS) (IAI: H4 904)

Course Meeting Times: Tuesdays 3:10-6:50

Times:

Course Meeting Times¶
PHIL 1700-07: Mon, Wed 3:10-4:25

PHIL 1700-08: Mon, Wed 4:35-5:50

Location¶
Xavier Hall 128

Course Pre-requisites¶
Area Prerequisites¶
D or better in CORE-1500

General Requirements¶
University Core 1500 Minimum Grade of D May not be taken concurrently.

Course Description¶
This course, PHIL 1700, invites students to explore enduring philosophical questions and to reflectively evaluate the various answers given them by thinkers from a range of social, historical, and religious contexts. Students will tackle ultimate questions in a range of philosophical domains, including issues such as the nature of self and ultimate reality, morality and human meaning, rationality and the pursuit of truth. The aim of the course is to give students an opportunity to critically examine their own beliefs and commitments in dialogue with each other and with great thinkers past and present. (Offered Fall, Spring, and Summer)

\hypertarget{instructor-information}{%
\chapter{Instructor Information}\label{instructor-information}}

Instructor: ``Montaque Reynolds'' ``\url{https://montaque-reynolds.com}''.

Contact: \href{mailto:montaque.reynolds@slu.edu}{\nolinkurl{montaque.reynolds@slu.edu}}

LMS: Messaging Service

\hypertarget{course-outcomes-objectives-and-competencies}{%
\chapter{Course Outcomes, Objectives, and Competencies}\label{course-outcomes-objectives-and-competencies}}

\hypertarget{outcomes}{%
\section{Outcomes}\label{outcomes}}

Students will be able to analyze written, oral, auditory, and visual messages and their implications in order to communicate effectively with a clear understanding
of audience, rhetorical purpose, argumentation, genre, and style.

Write and defend one's own \emph{and} another's ethical position clearly and concisely. Articulate the difference between ethics and metaethics. Approach a moral dilema from several varying ethical positions. Articulate their strengths and weaknesses.

Recognize the difference between persuasion and rhetoric.

\hypertarget{objectives}{%
\section{Objectives}\label{objectives}}

Critical thinking---from the scientific method to the creative process, from
systems thinking to complex abstractions---is a hallmark of a well-developed
mind.

Understand the difference between implicit and explicit premise, conclusion argumentation.

This course will introduce students to a variety of arguments and will ask them to critically reflect on these for systemic inquiry
and innovation.

\hypertarget{competencies}{%
\section{Competencies}\label{competencies}}

Understand the difference between implicit and explicit premise, conclusion argumentation.

Write and defend one's own \emph{and} another's ethical position clearly and concisely. Articulate the difference between ethics and metaethics. Approach a moral dilema from several varying ethical positions. Articulate their strengths and weaknesses.

Recognize the difference between persuasion and rhetoric.

\hypertarget{required-materials-and-equipment}{%
\chapter{Required Materials and Equipment}\label{required-materials-and-equipment}}

\hypertarget{required-text-book}{%
\section{Required Text Book}\label{required-text-book}}

Ethics: Theory and Contemporary Issues, MacKinnon, Wadsworth, latest edition.

\hypertarget{some-helpful-resources}{%
\section{Some Helpful Resources}\label{some-helpful-resources}}

\textbf{PhilPapers:} \url{https://philpapers.org/}

Provides a helpful overview of the field. It maps the relationships between various philosophers and disciplines within philosophy.

\textbf{Stanford (online) encyclopedia of philosophy:} \url{https://plato.stanford.edu/}

My search algorithm is likely going to be vastly different than yours. However, if you search up almost any philosophical question followed by ``sep'', you may likely find a link to an article on this page with an extensive development and critical reflection of that question.

\textbf{Wi-Phi}:

Many of our videos will come from here

\url{https://www.wi-phi.com/videos/}

\textbf{All-Sides:} \url{https://www.allsides.com/unbiased-balanced-news}

If you have ever worried about media bias, this website does a good job of scoring various publications according to any potential bias. This will be helpful since we will be looking at some articles about contentious issues and consider whether we are influenced by the truth of a claim, or by persuasive speech.

\hypertarget{evaluation-and-grading}{%
\chapter{Evaluation and Grading}\label{evaluation-and-grading}}

Much of the learning that you do in this course will be done at home. What this means, is that instead of applying what have learned in the classroom at home, working on a problem, writing a paper, etc., you will instead apply what you have learned at home, in the classroom.

\hypertarget{attendance-policy}{%
\chapter{Attendance Policy}\label{attendance-policy}}

\hypertarget{publishing}{%
\section{Publishing}\label{publishing}}

HTML books can be published online, see: \url{https://bookdown.org/yihui/bookdown/publishing.html}

\hypertarget{pages}{%
\section{404 pages}\label{pages}}

By default, users will be directed to a 404 page if they try to access a webpage that cannot be found. If you'd like to customize your 404 page instead of using the default, you may add either a \texttt{\_404.Rmd} or \texttt{\_404.md} file to your project root and use code and/or Markdown syntax.

\hypertarget{metadata-for-sharing}{%
\section{Metadata for sharing}\label{metadata-for-sharing}}

Bookdown HTML books will provide HTML metadata for social sharing on platforms like Twitter, Facebook, and LinkedIn, using information you provide in the \texttt{index.Rmd} YAML. To setup, set the \texttt{url} for your book and the path to your \texttt{cover-image} file. Your book's \texttt{title} and \texttt{description} are also used.

This \texttt{gitbook} uses the same social sharing data across all chapters in your book- all links shared will look the same.

Specify your book's source repository on GitHub using the \texttt{edit} key under the configuration options in the \texttt{\_output.yml} file, which allows users to suggest an edit by linking to a chapter's source file.

Read more about the features of this output format here:

\url{https://pkgs.rstudio.com/bookdown/reference/gitbook.html}

Or use:

\begin{Shaded}
\begin{Highlighting}[]
\NormalTok{?bookdown}\OperatorTok{::}\NormalTok{gitbook}
\end{Highlighting}
\end{Shaded}

\hypertarget{academic-honesty}{%
\chapter{Academic Honesty}\label{academic-honesty}}

\hypertarget{learning-accommodations}{%
\chapter{Learning Accommodations}\label{learning-accommodations}}

\hypertarget{title-ix}{%
\chapter{Title IX}\label{title-ix}}

\hypertarget{policies-on-generative-ai}{%
\chapter{Policies on Generative AI}\label{policies-on-generative-ai}}

  \bibliography{book.bib,packages.bib}

\end{document}
